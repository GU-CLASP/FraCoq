% 12/1/94, modified from book-hacks file

%**********************************
% self-comments and self-footnotes
% OFF ( 0 ) --- for ON ( 1 )
%**********************************
\newcommand{\selfcomment}[1]{\ifodd 0 {\sf #1 }\fi}
\newcommand{\selfcc}[1]{\ifodd 1 {\sf #1 }\fi}
\newcommand{\selfc}{\selfcomment}
\newcommand{\selffootnote}[1]{\ifodd 0 \footnote{{\sf #1}} \fi}
\newcommand{\selff}{\selffootnote}

%***************************************
% environments (sections, theorems, ...)
%***************************************
% sections etc.
%**************
\newcommand{\sect}{\section}
\newcommand{\subsect}{\subsection}
\newcommand{\ssubsect}{\subsubsection}
\newcommand{\para}{\paragraph}
\newcommand{\subpara}{\subparagraph}
%**************************
% theorems, definitions ...
%**************************
    
%**********************************************
% proof, convention, remark, notation, exercise
%**********************************************
%\newcommand{\proof}{{\bf Proof}\ }
\newcommand{\proofsketch}{{\bf Proof\ sketch}\ }
\newcommand{\convention}{\ \\{\bf Convention}\ }
\newcommand{\conventions}{\ \\{\bf Conventions}\ }
\newcommand{\notation}{\ \\{\bf Notation}\ }
\newcommand{\notations}{\ \\{\bf Notations}\ }
%\newcommand{\remark}{\ \\{\bf Remark}\ }
\newcommand{\remarks}{\ \\{\bf Remarks}\ }
%\newcommand{\exercise}{\ \\{\bf Exercise}\ }
%***************
% \qed and \blob
%***************
\newcommand{\blob}{\mbox{\rule[-1.5pt]{5pt}{10.5pt}}}
%\newcommand{\qed}{\quad\hfill\mbox{$\Box$}}

%****************************************************************
% Roman in Theorem env (\mbox{\em ...}) (for boldface words etc.)
%****************************************************************
\newcommand{\them}[1]{{\mbox{\em #1}}}

%***************
% cases in line
%***************
\newcommand{\casestwo}[2]{\cases {#1 \cr #2}}

%*******************************
% sets, sequences, and ECC-pairs
%*******************************
\newcommand{\set}[2]{\{\,#1\,\mid\,#2\,\}}      % set of #1 such that #2
\newcommand{\emptyseq}{\langle\rangle}          %<>  in math mode
% sequence of #1's, numbered up to #2  (e.g., seq{x}{n} for x1 ... xn   )
\newcommand{\seq}[2]{#1_{1} \ldots #1_{#2}}
% comma--separated sequence
\newcommand{\cseq}[2]{#1_{1},\ldots,#1_{#2}}
% ECC-pairs
\newcommand{\pair}[3]{\langle#1,#2\rangle_{#3}} % pairs in ECC

%**************************
% fix up colon conventions
%**************************
\mathcode`:="003A               % ordinary colon (tight spacing)
\mathchardef\colon="303A        % relation colon, rule colon
% \mathchardef\punctcolon="603A % colon as a punctuation mark

%***************
% abbreviations
%***************
%**************************
% math symbol abbreviations
%**************************
\newcommand{\cross}{\times}                     % cross
\newcommand{\del}{\delta}
\newcommand{\e}{\epsilon}
\newcommand{\g}{\gamma}
\newcommand{\G}{\Gamma}
\newcommand{\defd}{\mathord{\downarrow}}        % ``is defined'' symbol
\newcommand{\undefd}{\mathord{\uparrow}}        % ``is undefined'' symbol
\newcommand{\ts}{\vdash}        % turnstile
\newcommand{\tseq}{\vdash^=}    % turnstile^=
\newcommand{\dts}{\models}      % double turnstyle for model sat. etc.
\newcommand{\rts}{{\ \|\!\!-}}  % vertical-double turnstyle for realizability
\newcommand{\bottom}{\perp}     % bottom or undefined symbol
\newcommand{\lam}{\lambda}
\newcommand{\fa}{\forall}                       % for all
\newcommand{\te}{\exists}                       % there exists
\newcommand{\To}{\Rightarrow}                   % imply -- double right arrow
\newcommand{\fppto}{\to_{FPP}}                  % arrow_FPP
\newcommand{\isoto}{\buildrel \cong \over \to}  % iso over arrow
\newcommand{\Iff}{\mathrel{\Leftrightarrow}}    % double arrow both directions
\renewcommand{\iff}{\Iff}                       % ibid.
\newcommand{\infinity}{\infty}          % infinity
\newcommand{\intersect}{\cap}           % intersection
\newcommand{\nt}{\buildrel . \over \to} % natural transformation
\newcommand{\pto}{\rightharpoonup}      % partial function space
\newcommand{\sig}{\sigma}
\newcommand{\union}{\cup}               % union
\newcommand{\w}{\omega}
\newcommand{\iso}{\cong}          % isomorphism
\newcommand{\conv}{\simeq}        % conversion
\newcommand{\red}{\rhd}           % reduction
\newcommand{\bconv}{\conv_\beta}  % beta-convertible
\newcommand{\bred}{\rhd_\beta}    % beta-reducible
\newcommand{\contract}{\leadsto}  % contraction schemes, contract to
\newcommand{\subs}{\subseteq}
\newcommand{\lekind}{\preceq}
\newcommand{\lesskind}{\prec}
\newcommand{\gekind}{\succeq}
\newcommand{\grkind}{\succ}
\newcommand{\eqkind}{\approx}
%**************
% semantics of
%**************
\newcommand{\dlb}{\lbrack\!\lbrack}
\newcommand{\ldb}{\dlb}
\newcommand{\drb}{\rbrack\!\rbrack}
\newcommand{\rdb}{\drb}
\newcommand{\sem}[1]{\mathop{\ldb #1\rdb}\nolimits}     % semantics of ...
%*************************************************************
% \them should be used for the following in theorems !!!!
%*************************************************************
\newcommand{\df}{{\ =_{\mbox{\scriptsize df }}}}                 % =df
\newcommand{\iffdf}{{\ \iff_{\mbox{\scriptsize df }}}}           % \iff_df
\newcommand{\redk}{{\mbox{\rm red$_{\mbox{\scriptsize\bf k}}$}}} % key reduct
\newcommand{\rednf}{{ \mbox{$\red_{\mbox{\scriptsize nf}}$ }}}   % red to nf
%*******************
% text abbreviations
%*******************
\newcommand{\CR}{Church-Rosser theorem} % Church-Rosser theorem
\newcommand{\IH}{induction hypothesis}  % IH for induction hypothesis
\newcommand{\ML}{Martin-L\"{o}f}        % Martin-Lof
\newcommand{\qn}{quasi-normal}          % qn for quasi-normal
\newcommand{\SN}{strongly normalisable} % SN abbr. strongly normalizable
%**********************************************************
% system names in Text, \them should be used in theorems !!!
%**********************************************************
\newcommand{\ECC}{\mbox{{\rm ECC}}}                     % ECC \bf-->\rm
\newcommand{\ECCn}{{\ECC$^n$}}                          % ECC^n
\newcommand{\ECCbn}{{\ECC$^{\mbox{\scriptsize\bf n}}$}} % ECC^\n
\newcommand{\SCC}{{$\Sigma${\mbox{CC}}$_\subset\ $}}    % SigCC_INC
\newcommand{\SCCinf}{{$\Sigma${\mbox{CC}}$_\subset^\infty\ $}}  % SigCC_INC^inf
\newcommand{\CCinf}{{{\mbox{CC}}$^\infty_\subset$}\ }           % CC_INC^inf
% \newcommand{\CCn}{{{\mbox{CC}}$^\n_\subset$}\ }               % CC^\n
\newcommand{\CCinfcoq}{{{\mbox{CC}}$^\infty$}}                  % CC^inf
\newcommand{\SCCn}{{$\Sigma${\mbox{CC}}$^n_\subset$}\ }         % SigCC_INC^n
%*************
% roman (\rm)
%*************
\newcommand{\Obj}{{\mbox{\rm Obj}}}             % objects of a cat


%*************************************************************
% italic (\it) (some changed to roman suggested by the editor)
%*************************************************************
\newcommand{\cf}{{cf.}}
\newcommand{\eg}{{e.g.}}
\newcommand{\etal}{{\it et~al.}}
\newcommand{\etc}{{etc.}}
\newcommand{\ie}{{i.e.}}
\newcommand{\vs}{{\it vs.}}
\newcommand{\wrt}{{w.r.t.}}
\newcommand{\ala}{\`{a} la}
%****************************************
% capital calligraphic in mathmode (\cal)
%****************************************
\newcommand{\D}{{\cal D}}               % cal D, degree of E-terms
\newcommand{\E}{{\cal E}}               % cal E, env
% \newcommand{\K}{{\cal K}}             % cal K, the set of kinds
\newcommand{\U}{{\cal U}}               % cal U, the set of kinds
\newcommand{\level}{{\cal L}}           % cal L --- levels of types
\newcommand{\lev}{\level}               % cal L --- levels of types
% \newcommand{\T}{{\cal T_\E}}          % cal T_E
% \newcommand{\Check}[2]{{\cal C}(#1;\ #2)}       % type-checking algorithm
\newcommand{\Terms}{{\cal T}}           % the set of (untyped) terms
%*************************************************************************
% boldface (\bf) --- in Text or Mathmode, \them should be used in theorems.
%*************************************************************************
\newcommand{\1}{{\mbox{\bf 1}}}         % terminal object, unit type
\newcommand{\back}{{\mbox{\bf back}}}   % back:M->PROP
\newcommand{\false}{{\mbox{\bf false}}} % false, logical
\newcommand{\inc}{{\mbox{\bf inc}}}     % inclusion
\newcommand{\jbf}{{\mbox{\bf j}}}       % bf j
\newcommand{\key}{{\mbox{\bf key}}}     % key term of a term
\newcommand{\M}{{\mbox{\bf M}}}         % cat of modest sets
\newcommand{\n}{{\mbox{\bf n}}}         % bf n
\newcommand{\pairbf}{{\mbox{\bf pair}}} % pairs in UTT
\newcommand{\PER}{{\mbox{\bf PER}}}     % cat of pers
\newcommand{\PROP}{{\mbox{\bf PROP}}}   % cat of `propositions'
\newcommand{\Set}{{\mbox{\bf Set}}}     % cat of sets
\newcommand{\true}{{\mbox{\bf true}}}   % true, logical
\newcommand{\valid}{{\mbox{\bf valid}}} % context validity
\newcommand{\wS}{{\mbox{$\w${-}{\bf Set}}}}     % omega-Set -- cat of w-sets
\newcommand{\Prf}{{\mbox{\bf Prf}}}     % Prf(P) --- proof types

%****************
% inference rules
%****************
% \Frule{upper}{down}{side-condition}{rule-name}
\newcommand{\Frule}[4]
{\[ \hbox to \columnwidth
    { \rlap{$#4$} \hfil $
      {{\displaystyle\strut #1}\over{\displaystyle\strut #2}}
      \quad\makebox[0pt][l]{\it #3} $
      \hfil }
\]}
% 2 rules in the same line
\newcommand{\Tworules}[6]
{\[ #3 \ \ \
    {{#1}\over{#2}}
    \ \ \ \ \ \ \ \ \ \
    #6 \ \ \
    {{#4}\over{#5}}
\]}
\newcommand{\tworule}[4]                % special for LF-rules picture
{\[ {{#1}\over{#2}}
    \ \
    {{#3}\over{#4}}
\]}
% 3 rules in the same line
\newcommand{\Threerules}[9]
{\[ #3 \
    {{#1}\over{#2}}
    \ \ \ \ \
    #6 \
    {{#4}\over{#5}}
    \ \ \ \ \
    #9 \
    {{#7}\over{#8}}
\]}
\newcommand{\threerule}[6]              % special for LF-rules picture
{\[ {{#1}\over{#2}}
    \ \ \
    {{#3}\over{#4}}
    \ \ \
    {{#5}\over{#6}}
\]}
% 4 rules in the same line
\newcommand{\Fourrules}[8]
{\[ \ \ \
    {{#1}\over{#2}}
    \ \ \ \ \
    \ \ \
    {{#3}\over{#4}}
    \ \ \ \ \
    \ \ \
    {{#5}\over{#6}}
    \ \ \ \ \
    \ \ \
    {{#7}\over{#8}}
\]}
% 1 rule `in the same line'
\newcommand{\onerule}[2]
{\[ {{#1}\over{#2}} \]}
\newcommand{\Onerule}[3]
{\[ #3 \ \ \
    {{#1}\over{#2}}
\]}
% rule in table
\newcommand{\tablerule}[2]
{\begin{tabular}{c}\vspace{.05in}
 ${\displaystyle\strut #1}\over{\displaystyle\strut #2}$ % \vspace{.05in}\\
 \end{tabular}
}

%********************************************
% special commands for spec and theory papers
%********************************************
%**********************
% specification chapter
%**********************
\newcommand{\tuple}[1]{\langle #1 \rangle}              % new, 7/5/93
\newcommand{\CirSigmaText}{\sum\!\!\!\!\!\!\circ\ }
\newcommand{\CirSigmaMath}{\sum\!\!\!\!\!\!\!\circ\ }
% \newcommand{\Dot}{\bullet}

\newcommand{\List}{List}
\newcommand{\nil}{nil}
\newcommand{\cons}{cons}
\newcommand{\N}{N}
\newcommand{\suc}{succ}                         % added 31/12/92
\newcommand{\rec}{Rec}
%\renewcommand{\tt}{\mbox{tt}}                  % \tt is used for \verb etc.
\newcommand{\ff}{\mbox{ff}}
\newcommand{\ifN}{\mbox{if}_N}

\newcommand{\refn}{\rho}
\newcommand{\refnto}{\Longrightarrow}
\newcommand{\refncon}[1]{\buildrel c\over\Longrightarrow_{#1}}
\newcommand{\refnc}{\refncon}
\newcommand{\refnsel}[1]{\buildrel s\over\Longrightarrow_{#1}}
\newcommand{\refnabs}[2]{\buildrel a\over\Longrightarrow^{#1}_{#2}}

\newcommand{\SPEC}{{\mbox{\bf SPEC}}}
\newcommand{\Spec}{{\mbox{\bf Spec}}}
\newcommand{\Str}{{\mbox{\bf Str}}}
\newcommand{\Ax}{{\mbox{\bf Ax}}}
\newcommand{\Mod}{{\mbox{\bf Mod}}}
\newcommand{\Sat}{{\mbox{\bf Sat}}}
\newcommand{\Setoid}{{\mbox{\bf Setoid}}}
\newcommand{\Stack}{{\mbox{\bf Stack}}}
\newcommand{\STACK}{{\mbox{\bf STACK}}}
\newcommand{\Array}{{\mbox{\bf Array}}}
\newcommand{\ARRAY}{{\mbox{\bf ARRAY}}}
\newcommand{\Elem}{{\mbox{\bf Elem}}}
\newcommand{\Cong}{{\mbox{\bf Cong}}}
\renewcommand{\Join}{{\mbox{\bf Join}}}
\newcommand{\Meet}{{\mbox{\bf Meet}}}
\newcommand{\Extend}{{\mbox{\bf Extend}}}
\newcommand{\Con}{{\mbox{\bf Con}}}
\newcommand{\Sel}{{\mbox{\bf Sel}}}
\newcommand{\Abs}{{\mbox{\bf Abs}}}
\newcommand{\derive}{{\mbox{\bf derive}}}
%***************
% theory chapter
%***************
\newcommand{\Thm}{{\mbox{\bf Thm}}}     % !!!
\newcommand{\Prfs}{{\mbox{\bf Prfs}}}   % !!!
\newcommand{\THEORY}{{\mbox{\bf THEORY}}}

%******************
% logical framework
%******************
\newcommand{\LF}{{\sc LF}}
% \newcommand{\R}{{\sc R}}
\newcommand{\kind}{{\mbox{\bf kind}}}
\newcommand{\Type}{{\mbox{\bf Type}}}
\newcommand{\Ebf}{{\mbox{\bf E}}}
\newcommand{\emptytype}{\emptyset}
\newcommand{\App}{{\mbox{\bf App}}}
\newcommand{\app}{{\mbox{\bf app}}}
\newcommand{\T}{{\mbox{\bf T}}}
\renewcommand{\t}{{\mbox{\bf t}}}

\newcommand{\SOL}{{SOL}}
\newcommand{\plt}{plt}
\newcommand{\lts}{\models}
\newcommand{\TSCH}{{\sc Tsch}}
\newcommand{\pltRule}[1]{(\them{{\sc #1}})}

\newcommand{\Types}{{\sc Types}}
\newcommand{\Sch}{{\sc Isch}}
\newcommand{\Pos}{{\sc Pos}}
\newcommand{\Ari}{{\sc Ari}}

\newcommand{\Mu}{{\cal M}}
\newcommand{\leschema}{<}



%***********************
% other special commands
%***********************
\newcommand{\p}{\ \ \ }         % spaces used for index files for thesis
\newcommand{\mathover}[2]{\buildrel \rm #1 \over #2}    % #1 over #2
