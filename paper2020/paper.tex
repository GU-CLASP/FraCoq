\documentclass[a4paper,11pt]{article}
\usepackage[hyperref]{naaclhlt2019}
\usepackage{natbib}


\aclfinalcopy % Uncomment this line for the final submission
%\def\aclpaperid{***} %  Enter the acl Paper ID here

%\setlength\titlebox{5cm}
% You can expand the titlebox if you need extra space
% to show all the authors. Please do not make the titlebox
% smaller than 5cm (the original size); we will check this
% in the camera-ready version and ask you to change it back.

%\newcommand\BibTeX{B{\sc ib}\TeX}


%\usepackage[backend=bibtex,citestyle=authoryear,style=authoryear,natbib]{biblatex}
% new style not compatible with biblatex
%\bibliography{../bbcl_lccm2018/bbcl_lccm2018.bib,iwcs2019}

\usepackage[verbose]{newunicodechar}

\title{Bayesian Inference Semantics: A Modelling System and  A Test Suite}
\author{Jean-Philippe Bernardy \qquad Stergios Chatzikyriakidis \\
  University of Gothenburg\\
  {\tt firstname.lastname@@gu.se} \\}

\date{}

\begin{document}
\maketitle

\section{Introduction}
Stergios
\section{Background}
\paragraph{GF}
If needed
\paragraph{Coq}
If needed
\paragraph{Temporal Semantics}
Stergios

\section{Our Compositional Semantics}

- See our pervious papers, (adjectives, adverbs, nouns, verbs, anaphora, etc.)

- Here we simply ignore these aspects and it is safe for the reader to
assume a montegovian semantics and assume that anaphoroi are resolved
as they should be.

\section{Our Treatment of time}

Our basic approach is to extend predicates with two time parameters,
one corresponding with the starting time of the action and one with
its stopping time.

walk(john,t0,t1)

\paragraph{time context}

The compositional semantics carry a current time reference (itself an
interval). This current interval propagates through the compositional
interpraetation. By default every interpretation passes the temporal
context to its components --- but some may act on it.

\paragraph{tenses and temporal adverbs}

Tenses are represented syntactically (in GF!) as an attribute of the
clause. In our representation we deal only with present and past
tenses --- we find that FraCas does not exercise (if a variant of the
past is used, the additional information is also carried by adverbs,
in a more specific way.)

If the temporal context was modified by an adverb (or an averbial
clause, such as ``before James swam''), then the tense is ignored.
%
Otherwise, if the tense is the present, then we use a simple
$(now,now)$ interval, with $now$ being a constant.
%
If the tense is the past, then we locally quantify over a time
interval $(t_0,t_1)$, such that $t_1 < now$.

\paragraph{time references and aspectual classes}

A common theme in the testsuite is reference to previous occurences of an event

P1	Smith left.
P2	Jones left.
P3	Smith left before Jones left.
Q 	Did Jones leave after Smith left?

We have to arrange that the two occurences of ``Jones left'', in P1
and P3 refer to the same time intervals.

For this purpose we postulate the unicity of action for certain timed propositions:

P-unicity : P t1 t2 → P t3 t4 → (t1 = t3) ∧ (t2 = t4)

P-unicity holds only if P is an \emph{achievement} or an \emph{actitivty}.

(The difference between actitivty and achievement is that for the
latter time intervals can be assumed to be of nil duration. However
this appears to play little role, if any, in our analysis.)

A contrario, if P is a \emph{stative}, then we get a time-interval subsumption property:

P-subsumption : [t3,t4] ⊆ [t1,t2] -> P t1 t2 -> P t3 t4

This principle is used for example below:

P1	Smith arrived in Paris on the 5th of May, 1995.
P2	Today is the 15th of May, 1995.
P3	She is still in Paris.
Q 	Was Smith in Paris on the 7th of May, 1995? 


\paragraph{(un)repeatable achievements}
Using unicity of action, we can refute 279:

279:
P1	Smith wrote a novel in 1991.
Q 	Did Smith write it in 1992?
H 	Smith wrote it in 1992.

∃x. novel(x)
∀t<NOW.  ∧ write(smith,x,t,t) [the scope for the existential is extended according to our previous analysis of anaphora]

However, the testsuite instructs that we should not be able to refute
280, with the justification that ``wrote a novel'' is a repeatable
accomplishment:

280:
P1	Smith wrote a novel in 1991.
Q 	Did Smith write a novel in 1992?
H 	Smith wrote a novel in 1992.

∀t<NOW. ∃x. novel(x) ∧ write(smith,x,t,t)

Our analysis does not need to separate these as special cases. Indeed,
even if $write(smith,x)$ is an activity and thus subject to unicity of
action, in (280), $x$ is quantified existentially; we have two
different actions: $write(smith,x) t1 t2$ and $write(smith,y) t3 t4$,
and thus we can't deduce equality of the interval $t1,t2$ and $t3,t4$;
in turn the hypothesis can't be refuted.

\paragraph{interval adverbs, point adverbs, universal adverbs}
Time adverbs, such as "Since", "always", "never", "in two years", "by
11 am", etc. introduce *local* quantification over a timespan.

time interval:
⟦``in 1992, s''⟧(_) = ∃t1,t2.  [t1,t2] ⊆ 1992, ⟦s⟧(t1,t2)

Universal:
⟦``always s''⟧(_) = ∀t1,t2.  [t1,t2] ⊆ 1992, ⟦s⟧(t1,t2)

relative time:
⟦``in two years, s''⟧(t0,tx) = ∃t2.  ⟦s⟧(t0+2years,t2)

\paragraph{action-modification verbs}

``start to'', ``finish'', etc.

Lexical semantics transform the interval (see start_VVTiming in Coq code)


\section{Results and evaluation}
JP
\paragraph{interaction with negation and group readings}

\section{Conclusions and Future Work}
Stergios
\section*{Acknowledgements}

The research reported in this paper was supported by grant 2014-39 from the
Swedish Research Council, which funds the Centre for Linguistic Theory and
Studies in Probability (CLASP) in the Department of Philosophy, Linguistics,
and Theory of Science at the University of Gothenburg. We are grateful to
our colleagues in CLASP for helpful discussion of some of the ideas presented
here. We also thank three anonymous reviewers for their useful comments on an
earlier draft of the paper.

\bibliographystyle{acl_natbib}

\end{document}
